\documentclass{article}

\usepackage[margin=0.5in,bottom=1in,footnotesep=1in]{geometry}

\usepackage{amsmath}


\usepackage{multicol}
\setlength{\columnsep}{1cm}
\usepackage[]{algorithm2e}

\usepackage{lipsum}% for dummy text
\usepackage[varg]{txfonts}
\usepackage{graphicx}
\usepackage{subcaption}
\usepackage{multirow}

\usepackage[font=small,labelfont={sf,bf}]{caption}


\usepackage{titlesec}
\titleformat{\section}{\fontfamily{phv}\fontsize{12}{15}\bfseries}{\thesection}{1em}{}
\titleformat{\subsection}{\fontfamily{phv}\fontsize{10}{15}\itshape}{\thesubsection}{1em}{}

\title{\textbf{FYS4150 Project 3: \\Numerical integration}}
\author{Marie Foss, Maria Hammerstr{{\o}}m}
\date{} % removes date from title

\begin{document}

\maketitle

\begin{abstract}
	\noindent \lipsum[1]
	\vspace*{2ex}
	
	\noindent \textbf{Github:} \textit{https://github.com/mariahammerstrom/Project3}
	\vspace*{2ex}
\end{abstract}



\begin{multicols}{2}

\section{Introduction}
In this project we will determine the ground state correlation energy between two electrons in an helium atom by calculating a six-dimensional integral that appears in many quantum mechanical applications. The methods we will use are Gauss-Legendre and Gauss-Laguerre quadrature and Monte-Carlo integration. 

We assume that the wave function of each electron can be modeled like the single-particle wave function of an electron in the hydrogen atom. The single-particle wave function  for an electron $i$ in the $1s$ state  is given in terms of a dimensionless variable    (the wave function is not properly normalized)

\begin{equation}
	{\bf r}_i =  x_i {\bf e}_x + y_i {\bf e}_y +z_i {\bf e}_z ,
\end{equation}
as

\begin{equation}
	\psi_{1s}({\bf r}_i)  =   e^{-\alpha r_i},
\end{equation}
where $\alpha$ is a parameter and 

\begin{equation}
	r_i = \sqrt{x_i^2+y_i^2+z_i^2}.
\end{equation}
We will fix $\alpha=2$, which should correspond to the charge of the helium atom $Z=2$. 

The ansatz for the wave function for two electrons is then given by the product of two 
so-called $1s$ wave functions as 

\begin{equation}
	\Psi({\bf r}_1,{\bf r}_2)  =   e^{-\alpha (r_1+r_2)}.
\end{equation}

The integral we need to solve is the quantum mechanical expectation value of the correlation energy between two electrons which repel each other via the classical Coulomb interaction, namely

\begin{equation}\label{eq:correlationenergy}
	\langle \frac{1}{|{\bf r}_1-{\bf r}_2|} \rangle = \int_{-\infty}^{\infty} d{\bf r}_1d{\bf r}_2  e^{-2\alpha (r_1+r_2)}\frac{1}{|{\bf r}_1-{\bf r}_2|}.
\end{equation}
This integral can be solved in closed form, which gives an answer of $5\pi^2/16^2$.


...


\section{Methods}

\subsection{Gauss-Legendre and Gauss-Laguerre}
Gaussian quadrature (hereafter GQ) is a method that solves integrals with excellent results, giving high precision for few integration points, compared to simpler integration methods such as Newton-Cotes quadrature. 

The basic idea behind all integration methods is to approximate the integral

\begin{equation}
	I = \int_a^b f(x) dx \approx \sum_{i = 1}^N \omega_i f(x),
\end{equation}
where $\omega$ are the weights and $x$ are the chosen mesh points. The theory behind GQ is to obtain an arbitrary weight $\omega$, which will not be equally spaced, through the use of orthogonal polynomials, namely Legende and Laguerre polynomials in this case. For GQ we thus make the approximation

\begin{equation}
	f(x) \approx P_{2N-1} (x),
\end{equation}
where $P_{2N-1} (x)$ is a polynomial of degree $2N-1$ with N mesh points. The mesh points are the zeros of the chosen orthogonal polynomial of order $N$, and the weights are determined from the inverse of a matrix defined by the orthogonal polynomials. Thus, GQ says that

\begin{equation}
	\int f(x) dx \approx \int P_{2N-1}(x) dx \approx \sum_{i = 1}^{N-1} P_{2N-1}(x_i) \omega_i.
\end{equation}
The \textbf{Legendre polynomials} are solutions to a differential equation arising in for example the solution of the \textit{angular dependence} of Schr\"{o}dinger's equation with spherically symmetric potentials such as the Coulomb potential. The Legendre polynomials are defined as

\begin{equation}
	L_k(x) = \frac{1}{2^k k!} \frac{d^k}{dx^k} (x^2 - 1)^k \textrm{ , } k = 0,1,2, \dots
\end{equation}
Similarly, the \textbf{Laguerre polynomials} are solutions to the differential equation arising in for example the solution of the \textit{radial} Schr\"{o}dinger's equation as described above. The Laguerre polynomials are defined as

\begin{equation}
	L_n(x) = e^x \frac{d^n}{dx^n} (x^n e^{-x}) \textrm{ , } n = 0,1,2, \dots
\end{equation}


\subsection{Monte-Carlo}
...


\section{Results}
...



\section{Conclusions}
...





\section{List of codes}

The codes developed for this project are:\\
...

\end{multicols}

\end{document}
