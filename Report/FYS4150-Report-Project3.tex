\documentclass{article}

\usepackage[margin=0.5in,bottom=1in,footnotesep=1in]{geometry}

\usepackage{amsmath}


\usepackage{multicol}
\setlength{\columnsep}{1cm}
\usepackage[]{algorithm2e}

\usepackage{lipsum}% for dummy text
\usepackage[varg]{txfonts}
\usepackage{graphicx}
\usepackage{subcaption}
\usepackage{multirow}

\usepackage[font=small,labelfont={sf,bf}]{caption}


\usepackage{titlesec}
\titleformat{\section}{\fontfamily{phv}\fontsize{12}{15}\bfseries}{\thesection}{1em}{}
\titleformat{\subsection}{\fontfamily{phv}\fontsize{10}{15}\itshape}{\thesubsection}{1em}{}
\titleformat{\subsubsection}{\fontfamily{phv}\fontsize{9}{15}\bfseries}{\thesubsubsection}{1em}{}


\title{\textbf{FYS4150 Project 3: \\Numerical integration}}
\author{Marie Foss, Maria Hammerstr{{\o}}m}
\date{} % removes date from title

\begin{document}

\maketitle

\begin{abstract}
	\noindent \lipsum[1]
	\vspace*{2ex}
	
	\noindent \textbf{Github:} \textit{https://github.com/mariahammerstrom/Project3}
	\vspace*{2ex}
\end{abstract}



%\begin{multicols}{2}

\section{Introduction}
In this project we will determine the ground state correlation energy between two electrons in an helium atom by calculating a six-dimensional integral that appears in many quantum mechanical applications. The methods we will use are Gauss-Legendre and Gauss-Laguerre quadrature and Monte-Carlo integration. 

We assume that the wave function of each electron can be modeled like the single-particle wave function of an electron in the hydrogen atom. The single-particle wave function  for an electron $i$ in the $1s$ state  is given in terms of a dimensionless variable    (the wave function is not properly normalized)

\begin{equation*}
	{\bf r}_i =  x_i {\bf e}_x + y_i {\bf e}_y +z_i {\bf e}_z ,
\end{equation*}
as

\begin{equation*}
	\psi_{1s}({\bf r}_i)  =   e^{-\alpha r_i},
\end{equation*}
where $\alpha$ is a parameter and 

\begin{equation*}
	r_i = \sqrt{x_i^2+y_i^2+z_i^2}.
\end{equation*}
We will fix $\alpha=2$, which should correspond to the charge of the helium atom $Z=2$. 

The ansatz for the wave function for two electrons is then given by the product of two 
so-called $1s$ wave functions as 

\begin{equation}
	\Psi({\bf r}_1,{\bf r}_2)  =   e^{-\alpha (r_1+r_2)}.
\end{equation}
The integral we need to solve is the quantum mechanical expectation value of the correlation energy between two electrons which repel each other via the classical Coulomb interaction, namely

\begin{equation}\label{eq:correlationenergy}
	\langle \frac{1}{|{\bf r}_1-{\bf r}_2|} \rangle = \int_{-\infty}^{\infty} \mathrm{d}{\bf r}_1\mathrm{d}{\bf r}_2  e^{-2\alpha (r_1+r_2)}\frac{1}{|{\bf r}_1-{\bf r}_2|}.
\end{equation}
This integral can be solved in closed form, which gives an answer of $5\pi^2/16^2 \approx 0.192765710958777$.



\section{Methods}

\subsection{Gaussian quadrature}
Gaussian quadrature (hereafter GQ) is a method that solves integrals with excellent results, giving high precision for few integration points compared to simpler integration methods such as Newton-Cotes quadrature. The basic idea behind the method is to approximate the given function by a polynomial of degree $2N-1$ so that we can calculate our integral in the following way:

\begin{equation}
	I = \int_a^b f(x) \mathrm{d}x = \int_a^b W(x) g(x) \mathrm{d}x \approx \sum_{i = 0}^{N-1} \omega_i f(x),
\end{equation}
where $\omega$ are the weights given by the weight function $W(x)$ and $x$ are the chosen mesh points. The theory behind GQ is to obtain an arbitrary weight $\omega$, which will not be equally spaced, through the use of orthogonal polynomials, such as Legende and Laguerre polynomials. 

%For GQ we thus make the approximation
%
%\begin{equation}
%	f(x) \approx P_{2N-1} (x),
%\end{equation}
%where $P_{2N-1} (x)$ is a polynomial of degree $2N-1$ with N mesh points. The mesh points are the zeros of the chosen orthogonal polynomial of order $N$, and the weights are determined from the inverse of a matrix defined by the orthogonal polynomials. Thus, GQ says that
%
%\begin{equation}
%	\int f(x) dx \approx \int P_{2N-1}(x) dx \approx \sum_{i = 1}^{N-1} P_{2N-1}(x_i) \omega_i.
%\end{equation}


\subsubsection{Gauss-Legendre quadrature}
The Legendre polynomials are solutions to a differential equation arising in for example the solution of the \textit{angular dependence} of Schr\"{o}dinger's equation with spherically symmetric potentials such as the Coulomb potential. The Legendre polynomials are defined as

\begin{equation}
	L_k(x) = \frac{1}{2^k k!} \frac{d^k}{dx^k} (x^2 - 1)^k \textrm{ , } k = 0,1,2, \dots
\end{equation}
Using \textbf{Gauss-Legendre quadrature} means solving an integral from $- \infty$ to $+ \infty$. This can however not be achieved numerically. To find out how we can substitute the infinite limits with finite limits, we plot the single-particle wave function shown in Fig.~\ref{fig:integrand}, which shows that the function $e^{-\alpha r_i}$ reaches zero at approximately $r_i \approx 3$. We therefore used the limits [-3,3].

\begin{center}
	\includegraphics[width=90mm]{integrand.png} 	
	\captionof{figure}{The single-particle wave function $e^{-\alpha r_i}$.}
	\label{fig:integrand}
\end{center}



\subsubsection{Gauss-Laguerre quadrature}
Similarly, the Laguerre polynomials are solutions to the differential equation arising in for example the solution of the \textit{radial} Schr\"{o}dinger's equation as described above. The Laguerre polynomials are defined as

\begin{equation}
	L_n(x) = e^x \frac{\mathrm{d}^n}{\mathrm{d}x^n} (x^n e^{-x}) \textrm{ , } n = 0,1,2, \dots
\end{equation}
Our integral in Eq. (\ref{eq:correlationenergy}) can be rewritten with spherical coordinates to better deal with the infinite limits. Then the integral in Eq. (\ref{eq:correlationenergy}) becomes:

\begin{equation}
	I = \int \int \int \int  \int \int    r_1^2 r_2^2 \frac{e^{- 2 \alpha (r_1 + r_2)}}{r_{12}} \sin \theta_1 \sin \theta_2 \mathrm{d}r_1  \mathrm{d}r_2 \mathrm{d}\theta_1 \mathrm{d}\theta_2 \mathrm{d}\phi_1\mathrm{d}\phi_2
\end{equation}
with

\begin{equation}\label{eq:r12}
	\frac{1}{r_{12}}= \frac{1}{\sqrt{r_1^2+r_2^2-2r_1r_2\cos(\beta)}},
\end{equation}
where

\begin{equation*}
	\cos(\beta) = \cos(\theta_1)\cos(\theta_2)+\sin(\theta_1)\sin(\theta_2)\cos(\phi_1-\phi_2)).
\end{equation*}
The radial part of the integral has limits [0,$\infty$) and we can use Laguerre polynomials with a weight function $W(x) = r^{\alpha} e^{-r}$. But first we need to do a variable change where we substitute $r_1 = u_1/2\alpha$ and $r_2 = u_2/2\alpha$, and $\mathrm{d}r_1 = \mathrm{d}u_1/2\alpha$ and $\mathrm{d}r_2 = \mathrm{d}u_2/2\alpha$. In our case $\alpha = 2$. Thus the integral becomes:

\begin{equation}
	I = \frac{1}{1024} \int_0^{\pi} \sin \theta_1 \sin \theta_2 \mathrm{d}\theta_1 \mathrm{d}\theta_2 
	\int_0^{2\pi} \mathrm{d}\phi_1\mathrm{d}\phi_2
	\int_0^{\infty}   \mathrm{d}u_1  \mathrm{d}u_2 u_1^2 u_2^2 \frac{e^{- (u_1 + u_2)}}{\sqrt{r_1^2+r_2^2-2r_1r_2\cos(\beta)}}  
\end{equation}



\subsection{Monte Carlo}
The Monte Carlo method is a statistical simulation method which can be used for systems that are described by their probability distribution functions (PDFs). The Monte Carlo method proceeds by random samplings from the PDF. The final result is taken as an average over the number of simulations, and multiplied with the Jacobi determinant of the change of variables.
\begin{equation}
	-l + 2lx_i,
\end{equation}
where $l$ is the limit. The Jacobi determinant for this is $(2l)^6$, as we have 6 variables.

We also want to run the Monte Carlo method by using importance sampling, as this should give better precision.
For this case we return to spherical coordinates,
\begin{equation}
	I = \int\limits_0^{\infty}\mathrm{d}r_1\mathrm{d}r_2\int\limits_0^{\pi}\mathrm{d}\theta_1\mathrm{d}\theta_2\int\limits_0^{2\pi}\mathrm{d}\phi_1\mathrm{d}\phi_2
	\frac{r_1^2r_2^2\sin(\theta_1)\sin(\theta_2)}{r_{12}}
\end{equation}, where $r_{12}$ is given in eq.~(\ref{eq:r12}).
For $r_i$ we want to use exponentially distributed random numbers.
%In the regular Monte Carlo method we use uniformly distributed random numbers, but when considering importance sampling we will instead use exponentially distributed random numbers.
%We will also switch to spherical coordinates.
This is done by changing variables as follows
\begin{align*}
	r_i &\rightarrow -\frac{1}{4}\ln(1-x_i) \\
	\theta_i &\rightarrow \pi x_i \\
	\phi_i &\rightarrow 2\pi x_i,
\end{align*}
where $x_i$ is a random number between 0 and 1.
The Jocobi determinant is in this case given by $(1/4)^2\times\pi^2\times(2\pi)^2 = \pi^4/4$.


Lastly, we calculate the variance, which is given by 
\begin{equation}\label{eq:variance}
	\sigma^2 = \int\limits_{-\infty}^{\infty}\mathrm{d}xP(x)(x-\mu)^2 \,= \,\,<x^2>-<x>^2.
\end{equation}


\section{Results}
The results for the various methods are collected here:

\begin{center}
\begin{tabular}{ l l l l l}\hline
	Method 									& N	 				&I			&$\sigma$				& $t$ (s) \\ \hline
	Gauss-Legendre 							& 25 					& 0.1958		& -						& 51 \\
	Gauss-Laguerre 							& 25					& 0.1917		& -						& 38 \\
	Monte Carlo 								& 5 $\cdot 10^7$ 		& 0.1972		& 7.00147 $\cdot 10^{-3}$		& 23 \\
	Monte Carlo (with importance sampling) 			& 1 $\cdot 10^5$		& 0.1959		& 1.45335 $\cdot 10^{-2}$		& 0 \\
	\hline
\end{tabular}
\end{center}
COMMENTS.



\section{Conclusions}
...





\section{List of codes}

The codes developed for this project are:\\
...

%\end{multicols}

\end{document}
